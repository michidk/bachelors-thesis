% !TeX root = ../main.tex
\chapter{\abstractname}

Virtual Reality is an emerging medium which enables presence and interactivity in a three-dimensional space. Common input devices like a mouse or a keyboard are made for two-dimensional environments. They can work in three-dimensional environments as well but are tedious to use and require complex movements to complete a task. Thus, new input methods have to be developed. 
% In Virtual Reality, the view is obscured by the head-mounted display, so physical devices need a virtual representation. Most consumer-ready Virtual Reality peripherals, therefore, require an expensive tracking system.

Most people own a smartphone, which they use on a daily basis. They have a variety of different sensors already built-in, wireless capabilities and are able to run custom software. This makes them affordable input devices. Thanks to the orientational sensors, a virtual representation of the phone can be displayed. Therefore they are suitable to use as interaction devices for Virtual Reality.

To verify that a smartphone can be used as an input device for Virtual Reality, three interaction examples are presented. A model viewing application, a pointing tool, and a virtual keyboard were implemented and evaluated. The UBI-Interact networking solution is used to make the system reusable and abstracted from device-specific environments.