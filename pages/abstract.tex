% !TeX root = ../main.tex
\chapter{\abstractname}

Virtual Reality is an emerging medium which enables immersive presence and interactivity in a three-dimensional space. Standard input devices like a mouse or keyboard are made for two-dimensional environments. Thus, new interaction devices have to be invented. This is not an easy task since the view is obscured by the headset, so all devices need a virtual representation. Most consumer-ready Virtual Reality peripherals, therefore, require an expensive tracking system.

Smartphones have a variety of different sensors built-in and also can run custom software. This makes them cheap general-purpose devices. Thanks to the orientational sensors and wireless capabilities, the phone can be represented in a virtual environment. Thus it is possible to use them as interaction devices for Virtual Reality.

To prove that the smartphone can be used as an input device for Virtual Reality, three interaction examples are presented. An orientation-based device, a pointing device and a virtual keyboard were implemented and evaluated in a SUS usability study. The UBI-Interact networking solution is used to make the system reusable and abstracted from device-specific environments.