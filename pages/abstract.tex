% !TeX root = ../main.tex
% important chapter: spell out acronyms 
\chapter{\abstractname}

\glsentrylong{VR} is an emerging medium that enables presence and interactivity in a three-dimensional space. Common input devices like a mouse or a keyboard are made for two-dimensional environments. They require complex movements to complete tasks in a three-dimensional environment.% They can work in three-dimensional environments as well but are tedious to use and require complex movements to complete a task. Thus, new input methods have to be developed. 
% In Virtual Reality, the view is obscured by the head-mounted display, so physical devices need a virtual representation. Most consumer-ready Virtual Reality peripherals, therefore, require an expensive tracking system.

Most people own a smartphone which they use on a daily basis. Such phones have a variety of different sensors already built-in, feature wireless capabilities and are able to run custom software. This makes them affordable general-purpose devices. A virtual representation of the phone can be displayed in a \glsentrylong{VE} using the orientational sensors. Therefore they are suitable to use as interaction devices for \glsentrylong{VR}.

In this thesis three interaction examples are presented to verify that a smartphone can be used as an input device for \glsentrylong{VR}. A model viewing application, a pointing tool, and a virtual keyboard were implemented and evaluated. 

The \glsentrylong{UBII} networking framework is used to make the proposed experiments reusable and abstracted from device-specific environments. It connects the devices together and provides an extensible protocol which was adjusted to the needs of the experiments presented in this thesis.