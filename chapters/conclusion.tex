% !TeX root = ../main.tex
% important chapter: spell out acronyms like in abstract
\chapter{Conclusion}\label{chapter:conclusion}

To show that the smartphone is a valuable device for interacting with \glsentrylong{VR}, typical input methods used in \glsentrylong{VR} were explored and evaluated. A \glsentrylong{SUS} study showed that all three experiments, the model viewer, the laser pointer, and the virtual keyboard experiment, were usable.s

Three-dimensional models can be viewed with the model viewer experiment. In the evaluation most participants agreed that this input method is intuitive to operate.

The laser pointer is used to select elements in a \glsentrylong{UI} or for similar pointing tasks. This experiment scored the highest amongst the ones presented in this thesis. 

The virtual keyboard experiment solves the problem of typing text while being immersed in a \glsentrylong{VE}. While the model viewer and the laser pointer scenario reached a high score, the virtual keyboard scored slightly lower. 

A lot of feedback was collected during the survey, which can be used to improve these implementations further. Since all implementations are considered \enquote{acceptable}, it can be assumed that the smartphone is indeed a helpful input device for \glsentrylong{VR}.

The implementation used the \glsentrylong{UBII} system to abstract parts of the application, which makes the system more modular and extensible. This was achieved by implementing logic into \enquote{Interactions}, which are processed on the server. 