% !TeX root = ../main.tex
\chapter{Conclusion}\label{chapter:conclusion}

To show that the smartphone is a valuable device for interacting with \gls{VR}, typical input methods used in \gls{VR} were explored and evaluated. A \gls{SUS} study showed that all three experiments were usable. 

three-dimensional models can be viewed, with the model viewer experiment. In the evaluation, most participants agreed that this input method is intuitive to operate.

The laser pointer is used to select elements in a \gls{UI} or for similar pointing tasks. This experiment scored the highest amongst the ones presented in this thesis. 

While the model viewer and the laser pointer scenario score a relatively high score, the virtual keyboard scored slightly lower. A lot of feedback was collected during the survey, which can be used to improve the implementations further.

Since all implementations are considered \enquote{acceptable}, it can be assumed that the smartphone is indeed a helpful input device for \gls{VR}.

The implementation used the \gls{UBII} system to abstract parts of the application to provide extensibility. This was achieved by implementing so-called \enquote{Interactions}, which are processed on the server. 