% !TeX root = ../main.tex
\chapter{Introduction}\label{chapter:introduction}

\section{Motivation}\label{section:motivation}

\gls{VR} is an emerging technology, which provides new ways to present and interact with digital information. \citeauthor{Sherman.2003} define \gls{VR} using the four key elements virtual world, immersion, sensory feedback, and interactivity. They define the virtual world as an imaginary space which may be manifested through a medium. It is also a description of objects in a space together with rules and relationships. According to the authors, immersion is the feeling of presence in a virtual world. An essential ingredient of \gls{VR} is the sensory feedback which describes the feedback the \gls{VR} system conveys to the users depending on the users' state in the virtual world. %In order for \gls{VR} to seem authentic, it should respond to the user's actions to make it interactive~\cite[6-13]{Sherman.2003}.
\gls{VR} should respond to the users' actions to make it interactive, in order for \gls{VR} to seem authentic~\cite[6-13]{Sherman.2003}.
These four elements form the definition as a medium composed of interactive computer simulations that may sense the users' behavior and replace or augment the sensory feedback, with the goal of immersing the users in a virtual world~\cite[13-14]{Sherman.2003}.

In order to immerse users in a \gls{VE}, a display device (\gls{HMD}) is required. Most \glspl{HMD} have to be connected to a \gls{PC}. Some \gls{VR} systems use a smartphone or similar technology as a display device in the \gls{HMD}. The \gls{PC}, which processes data from input devices like motion controllers or motion data from the \gls{HMD}, is required for most consumer \gls{VR} systems. Often an external tracking system is also required. An application rendering three-dimensional content to the \gls{HMD} is necessary to present the \gls{VE} to the users. This works similar to a computer game rendering to a regular display but is more complicated.

While \gls{VR} can be used to experience all kinds of exciting and useful virtual worlds, it shines when interactivity comes into play. Since consumer \glspl{HMD} are now available, the development of tracked hand controllers (also known as \gls{VR}, three-dimensional, hand or motion controllers) is becoming more important.
Best practices are not yet defined, which leaves much room for new methods and research. Figure~\ref{fig:vr-controllers} illustrates the variety of different consumer \gls{VR} controllers available.

\begin{figure}[H]%
	\centering%
	\includegraphics[width=10cm]{figures/introduction/vr_controllers.jpg}%
  \caption[Collection of VR controllers]{A collection of different \gls{VR} controllers. From left to right, top to bottom: HTC VIVE Controllers, Valve Index Controllers (\enquote{Knuckles}), VIVE Tracker, Oculus Touch Controllers, Samsung Odyssey Controllers.
  \newline{}
  Source:~\cite{Yang.2018}}\label{fig:vr-controllers}
\end{figure}

One of the best-known motion controllers came with the gaming console Wii\footnote{The Wii is a gaming console developed by Nintendo. Website: \href{https://www.nintendo.co.uk/Wii/Wii-94559.html}{www.nintendo.co.uk/Wii}}. It can track the position and orientation of the users' hands, which made it easier to immerse the user in the game world~\cite[2]{Kamm.2018}.

Mapping the movement of the users' real hands to the virtual world is also a common strategy in current \gls{VR} hardware. Not only does it enhance the virtual presence by showing users a representation of their body, but it also gives users a natural way of controlling and interacting with the virtual world.

The Leap Motion\footnote{The Leap Motion controller is a hand tracking device, which is often used to display a hand avatar. Website: \href{https://www.leapmotion.com/}{www.leapmotion.com}} sensor uses multiple infrared cameras to track hand poses, which is only possible in front of the sensor. Newer generations of \gls{VR} controllers try to achieve a similar effect with different methods: The Oculus Touch\footnote{The Oculus Touch controllers are hand tracking devices included with the Oculus Rift \gls{HMD}. Website: \href{https://www.oculus.com/rift/}{www.oculus.com/rift}} controllers track the distance of the fingers from the controller and the Valve Index\footnote{The Valve Index is a \gls{HMD} which includes its own set of controllers, called \enquote{Knuckles}. Website: \href{https://store.steampowered.com/valveindex}{store.steampowered.com/valveindex}} controllers even have pressure sensors built-in.

However, for many interactions, hand inspired controllers are not ideal~\cite{Bowman.2012}. This applies especially to productive \gls{VR} applications, which require interactions like inputting text for labeling or manipulating three-dimensional shapes. Most \gls{VR} controllers also require complex and costly tracking systems.


The Google Cardboard\footnote{The Google Cardboard is a \gls{HMD} made out of cardboard, which uses a smartphone as a display and for tracking. Website: \href{https://vr.google.com/cardboard/}{vr.google.com/cardboard}} uses a smartphone as a display and as a tracking device. This demonstrates the versatility of smartphones. Most people already have one since smartphones are portable general-purpose devices which are not expensive anymore. Thanks to \glspl{WLAN} and Bluetooth\footnote{Bluetooth is a wireless standard for exchanging data over short ranges between mobile devices.} it is easy to connect the smartphone to other devices.

Smartphones have input devices like buttons, a touch screen and an \gls{IMU}\footnote{An IMU is an electronic component which is part of most smartphones and allows to measure the acceleration and the magnetic field. This allows tracking the rotation of a device in three dimensions~\cite[139]{Katzakis.2010}.}. Also output devices like the display, vibration motors, and speakers are built-in. This makes them comparable to \gls{VR} controllers which have similar components.

One significant difference between smartphones and common \gls{VR} controllers is that smartphones are not capable of accurate positional tracking. The position can be estimated using the data of an \gls{IMU}, but since the error accumulates over time~\cite[44]{Steed.2013}, this method cannot be used. Additional tracking methods, like using the \gls{WLAN} signal strength, can be used to correct the drift~\cite{Zhang.2015}. However, those methods are still not good enough, because \gls{VR} requires very accurate tracking with short distances.
Apart from the missing positional tracking, the other advantages lead to the assumption that the smartphone can be used as an alternative \gls{VR} controller.


\section{Problem Statement}\label{section:problem-statement}
This thesis aims to explore the possibilities of using the smartphone as an interaction device in \gls{VR} experiences. The fundamental question is, whether smartphones are useable as \gls{VR} input devices.

% were implemented instead of was im
To answer this question, some promising typical \gls{VR} interaction methods were implemented using a smartphone. The goal of those experiments is not to create a better system, but rather to show that the smartphone is equally capable of specific interactions as common \gls{VR} controllers.

To benchmark the performance, a user study was performed where participants complete tasks using the prosed input systems.
The performance of the users in those tasks was evaluated and if possible, compared with similar methods from other research.

Additionally, a \gls{SUS} user study was performed to get an assessment of the users' feel for the interface.
The \gls{UBII} system, a networking solution for distributed systems, was used to implement an abstracted and reusable system.


\section{Outline}\label{section:outline}
In Chapter~\ref{chapter:related-work}, different input methods using a smartphone or similar devices from previous research are highlighted. The \gls{UBII} components and architecture, as well as the web-based technology stack used in this project, is then introduced and broken down in Chapter~\ref{chapter:implementation}. Following this, Chapter~\ref{chapter:experiments} introduces different methods of using the smartphone as an alternative input device for typical \gls{VR} interactions. In Chapter~\ref{chapter:evaluation}, tasks to benchmark the users' performance are described. Subsequently, the user study and its results are presented. Following the evaluation of the user study results, the possibilities of future research are explored in Chapter~\ref{chapter:future-work}. This thesis ends, with a conclusion being drawn in Chapter~\ref{chapter:conclusion}.