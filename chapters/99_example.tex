% !TeX root = ../main.tex
% Add the above to each chapter to make compiling the PDF easier in some editors.

\chapter{Introduction}\label{chapter:asdf}

\section{Section}


\subsection{Cite Tests}
Citation test~\parencite{Afonso.2017} vs~\cite{Afonso.2017}.\\
More~\cite[pre][post]{Afonso.2017}\\
Also possible~\footcite{Afonso.2017}\\
\\
Acronym test \ac{VR} cool.\\
And a second time \ac{VR} awesome.\\

\tikzstyle{box}=[
  rectangle,
  rectangle split, 
  rectangle split draw splits=false,
  rounded corners,
  minimum height=0.5cm,
  inner sep=4pt,
  text centered,
]

\subsection{Draw Test}
\begin{figure}[htpb]
	\begin{tikzpicture}

		\node[box, draw=TUMBlue, very thick, text width=5cm, rectangle split parts=3] at (0,0) {
      \Large\textbf{Client}\normalsize

      \nodepart{two}
      Network device
      
      \nodepart{three}
			\tikz\node[box, draw=TUMSecondaryBlue, thick, text width=4.5cm, rectangle split parts=5] {
        \Large\textbf{Device}\normalsize
  
        \nodepart{two}
        Group of components
        
        \nodepart{three}
        \tikz\node[box, draw=TUMSecondaryBlue2, thick, text width=4cm, rectangle split parts=2] {
          \Large\textbf{Component}\normalsize
    
          \nodepart{two}
          \begin{itemize}[noitemsep, topsep=0pt, label={--}]
            \item Topic
            \item Message Format
            \item Input or Output
          \end{itemize}
        };
        \nodepart{four}
        \tikz\node[box, draw=TUMSecondaryBlue2, thick, text width=4cm, rectangle split parts=2] {
          \Large\textbf{Component}\normalsize
    
          \nodepart{two}
          \begin{itemize}[noitemsep, topsep=0pt, label={--}]
            \item Topic
            \item Message Format
            \item Input or Output
          \end{itemize}
        };
        \nodepart{five}
        \tikz\node[box, draw=TUMSecondaryBlue2, thick, text width=4cm, rectangle split parts=1] {
          \Large\textbf{. . .}\normalsize     
        };
			};
    };
    
    \node[box, draw=TUMBlue, very thick, text width=7cm, rectangle split parts=4] (session) at (8.5,0) {
      \Large\textbf{Session}\normalsize
  
      \nodepart{two}
      Server-side specification
      
      \nodepart{three}
			\tikz\node[box, draw=TUMSecondaryBlue, thick, text width=4.5cm, rectangle split parts=5] {
        \Large\textbf{Interaction}\normalsize
  
        \nodepart{two}
        Group of components
        
        
      };

      \nodepart{four}
			\tikz\node[box, draw=TUMSecondaryBlue, thick, text width=4.5cm, rectangle split parts=5] {
        \Large\textbf{Mappings}\normalsize
  
        \nodepart{two}
        Maps topics to the interaction inputs/outputs.
			};
    };

    \node[box, draw=TUMBlue, very thick, text width=7cm, rectangle split parts=2, below of = session, node distance=5cm] {
      \Large\textbf{Topic}\normalsize
  
      \nodepart{two}
      Topic list and topic data
    };
		
	\end{tikzpicture}
\caption{Do not forget!}\label{fig:sample-drawing2}
\end{figure}

\subsection{Other Tests}

See~\autoref{tab:sample}, \autoref{fig:sample-drawing}, \autoref{fig:sample-plot}, \autoref{fig:sample-listing}.

\begin{table}[htpb]
  \caption[Example table]{An example for a simple table.}\label{tab:sample}
  \centering
  \begin{tabular}{l l l l}
    \toprule
      A & B & C & D \\
    \midrule
      1 & 2 & 1 & 2 \\
      2 & 3 & 2 & 3 \\
    \bottomrule
  \end{tabular}
\end{table}

\begin{figure}[htpb]
  \centering
  % This should probably go into a file in figures/
  \begin{tikzpicture}[node distance=3cm]
    \node (R0) {$R_1$};
    \node (R1) [right of=R0] {$R_2$};
    \node (R2) [below of=R1] {$R_4$};
    \node (R3) [below of=R0] {$R_3$};
    \node (R4) [right of=R1] {$R_5$};

    \path[every node]
      (R0) edge (R1)
      (R0) edge (R3)
      (R3) edge (R2)
      (R2) edge (R1)
      (R1) edge (R4);
  \end{tikzpicture}
  \caption[Example drawing]{An example for a simple drawing.}\label{fig:sample-drawing}
\end{figure}

\begin{figure}[htpb]
  \centering

  \pgfplotstableset{col sep=&, row sep=\\}
  % This should probably go into a file in data/
  \pgfplotstableread{
    a & b    \\
    1 & 1000 \\
    2 & 1500 \\
    3 & 1600 \\
  }\exampleA{}
  \pgfplotstableread{
    a & b    \\
    1 & 1200 \\
    2 & 800 \\
    3 & 1400 \\
  }\exampleB{}
  % This should probably go into a file in figures/
  \begin{tikzpicture}
    \begin{axis}[
        ymin=0,
        legend style={legend pos=south east},
        grid,
        thick,
        ylabel=Y,
        xlabel=X
      ]
      \addplot table[x=a, y=b]{\exampleA};
      \addlegendentry{Example A};
      \addplot table[x=a, y=b]{\exampleB};
      \addlegendentry{Example B};
    \end{axis}
  \end{tikzpicture}
  \caption[Example plot]{An example for a simple plot.}\label{fig:sample-plot}
\end{figure}

\begin{figure}[htpb]
  \centering
  \begin{tabular}{c}
  \begin{lstlisting}[language=SQL]
    SELECT * FROM tbl WHERE tbl.str = "str"
  \end{lstlisting}
  \end{tabular}
  \caption[Example listing]{An example for a source code listing.}\label{fig:sample-listing}
\end{figure}
