% !TeX root = ../../main.tex
\section{Overview}\label{section:evaluation-overview}

To show that the smartphone is a suitable interaction device in use with \ac{VR}, the results of the usability study of the three experiments are presented. Simple tasks were implemented into the experiments, to measure the usability. After completing the task, they have to answer \ac{SUS} questions.

The general procedure of the experiment is as follows:
\begin{enumerate}
  \item Introduce the user to the topic
  \item Let the user fill out the consent form
  \item Randomly choose an order for the experiments
  \item Hand the user the \ac{HMD} and Smartphone
  \item For each experiment:
  \begin{enumerate}
  \item Brief the user for the experiment
  \item Let the user play around in the experiment
  \item Start the task as soon as the user feels confident with the interactions
  \item Save the anonymized task results
  \item Question the user the usability questions
  \end{enumerate}
\end{enumerate}

The evaluation was conducted in two different locations. Before starting the study, the \ac{WLAN} connection and network performance were tested and evaluated as appropriate. The specifications of the \ac{PC}, \ac{HMD} and Smartphone of the different evaluation setups, is listed in the Appendix~\ref{chapter:append-user-eval-devices}. The \ac{PC} was able to run the application smoothly with an average of 60 frames per second, which is enough to run a smooth \ac{VR} experience. The \ac{WLAN} connection and devices were capable of 20 Mbps\footnote{Mbps stands for megabits per second. This unit is often used in reference to internet speeds.}, which is enough for synchronizing data without a visible lag.

After each experiment, a \ac{SUS} survey was conducted. The \acf{SUS} uses a set of 10 questions, which are rated from strongly disagree (1) to strongly agree (5), to assess the usability of a system~\cite[3]{Brooke.1996}. \citeauthor{Finstad.2006}'s suggestions to change the eighth question to make it easier understandable for non-native speakers was implemented~\cite[188]{Finstad.2006}.