% !TeX root = ../../main.tex
\chapter{Evaluation}\label{chapter:evaluation}

To show that the smartphone is a suitable interaction device in use with \ac{VR}, the results of the usability study of the three experiments are presented. Simple tasks were implemented into the experiments, to measure the usability. After completing the task, they have to answer \ac{SUS} questions.

The general procedure of the experiment is as follows:
\begin{enumerate}
  \item Introduce the user to the topic
  \item Let the user fill out the consent form
  \item Randomly choose an order for the experiments
  \item For each experiment:
  \begin{enumerate}
  \item Brief the user for the experiment
  \item Let the user play around in the experiment
  \item Start the task as soon as the user feels confident with the interactions
  \item Save the anonymized task results
  \item Question the user the usability questions
  \end{enumerate}
\end{enumerate}

For the evaluation, two devices, connected to the same \ac{WLAN}, were used:
\begin{itemize}
  \item \ac{PC}
  \begin{itemize}
    \item Operating System: Windows 10
    \item RAM: 32 GB
    \item CPU: Intel(R) Core(TM) i7-6700K % chktex 36 chktex 8
    \item GPU: NVIDIA GeForce GTX 1080  
  \end{itemize}
  \item Smartphone
  \begin{itemize}
    \item Operating System: Android 9
    \item RAM: 8 GB
    \item CPU: Snapdragon(TM) 845 % chktex 36
  \end{itemize}
\end{itemize}

The \ac{PC} was able to run the application smoothly with an average of 60 frames per seconds, which is enough to run a smooth \ac{VR} experience. The \ac{WLAN} connection and devices were capable of 20 Mbps, which is enough for synchronizing data without a visible lag.