% !TeX root = ../../main.tex
\chapter{Evaluation}\label{chapter:evaluation}

The goal of the evaluation is not to compare the presented methods to previous ones, but rather show that the smartphone is a usable suitable interaction device in use with \ac{VR}. 
To measure the usability of the experiments, simple tasks were implemented.

The general procedure of the experiment is as follows:
\begin{enumerate}
  \item Introduce the user to the topic
  \item Let the user fill out the consent form
  \item Randomly choose an order for the experiments
  \item For each experiment:
  \begin{enumerate}
  \item Brief the user for the experiment
  \item Let the user play around in the \ac{VE} of the experiment
  \item Start the task as soon as the user feels confident with the interactions
  \item Save the anonymized task results
  \item Question the user the usability questions
  \end{enumerate}
\end{enumerate}

For the evaluation, two devices were used:
\begin{itemize}
  \item Personal Computer
  \begin{itemize}
    \item Operating System: Windows 10
    \item RAM: 32 GB
    \item CPU: Intel(R) Core(TM) i7-6700K CPU @ 4.00GHz % chktex 36 chktex 8
    \item GPU: NVIDIA GeForce GTX 1080  
  \end{itemize}
  \item Smartphone
  \begin{itemize}
    \item Operating System: Android 9
    \item RAM: 8 GB
    \item CPU: Snapdragon(TM) 845
  \end{itemize}
\end{itemize}

The Personal Computer was able to run the experience smoothly with an average of 60 frames per seconds.